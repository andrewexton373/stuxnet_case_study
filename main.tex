% Term paper proposal - Andrew Exton
% CSC 300: Professional Responsibilities
% Dr. Clark Turner

% One Column Format
\documentclass[12pt]{article}

\usepackage{setspace}
\usepackage{url}
\usepackage{multicol}
\usepackage{footmisc}
\usepackage{titlesec}
\usepackage{framed}
\usepackage{csquotes}
\usepackage{soul}


%%% INDENTATION DEPTH
\setcounter{secnumdepth}{5}
\setcounter{tocdepth}{5}

%%% PARAGRAPH TITLE FORMAT
\titleformat{\paragraph}
{\normalfont\normalsize\bfseries}{\theparagraph}{1em}{}
\titlespacing*{\paragraph}
{0pt}{3.25ex plus 1ex minus .2ex}{1.5ex plus .2ex}

%%% SUBPARAGRAPH TITLE FORMAT
\titleformat{\subparagraph}
{\normalfont\normalsize\bfseries}{\thesubparagraph}{1em}{}
\titlespacing*{\subparagraph}
{0pt}{3.25ex plus 1ex minus .2ex}{1.5ex plus .2ex}

%%% PAGE DIMENSIONS
\usepackage{geometry} % to change the page dimensions
\geometry{letterpaper}


\begin{document}

\title{\vfill Stalling Iranian Uranium Enrichment: a Stuxnet Case Study } %\vfill gives us the black space at the top of the page
\author{
By Andrew Exton \vspace{10pt} \\
CPE 300: Professional Responsibilities  \vspace{10pt} \\
Dr. Clark Turner \vspace{10pt} \\
}
\date{\today} %Or use \today for today's Date

\maketitle

\vfill  %in combination with \newpage this forces the abstract to the bottom of the page
\begin{abstract}

In 2010, a computer virus purportedly developed by the United States and Israel, destroyed Iranian uranium enrichment centrifuges.\cite{theRealStoryOfStuxnet} Was it ethical for these state sponsors to develop and deploy this weaponized virus as a means of limiting Iran’s nuclear capabilities?

Some argue it protected the global community by limiting Iran's ability to pursue nuclear weapon development.\cite{theRealStoryOfStuxnet} Others suggest this action restricted Iran's sovereign right to pursue nuclear technology as permitted by the Nuclear Non-Proliferation Treaty.\cite{lookIntoIranianNuclearProgram} By applying ethical principles from tenets 3.13 and 1.01 of the IEEE/ACM Software Engineering Code of Ethics and Professional Practice, we determine that this use of a state-sponsored cyber-weapon is unethical because it produced inaccurate data, derived unethically without proper authorization.  In addition, the state sponsors have not accepted full responsibility for their actions.

%%% MAIN CODE TENET + PLAIN LANGUAGE REASON !!!!!

\end{abstract}

\thispagestyle{empty} %remove page number from title page, but still keep it as pg #1
\newpage

\tableofcontents
\newpage

\begin{multicols}{2}

%%%%%%%%%%%%%%%%%%%%
%%% Known Facts  %%%
%%%%%%%%%%%%%%%%%%%%
\section{Facts}

Throughout the early 2000s, the Islamic Republic of Iran's nuclear program was a topic of grave concern and sparked debate in forums such as the United Nations (UN).\cite{unitedNationsResolutions} The UN imposed international mandates on Iran, requiring the disclosure of Iran's nuclear program activity to the International Atomic Energy Association (IAEA), following state policy statements targeting their nuclear weapon program advancements. 

After reviewing Iran's October 21st, 2003 Nuclear Program Declaration, the IAEA questioned why certain centrifuge designs uncovered during the inspection were not included in the report, which Iranian authorities responded they, "neglected to include due to time considerations".\cite{implementationOfNPTSafeguards}

In response to Iran's enrichment activity, which violated provisions of previous UN Resolutions, the United Nations Security Council (UNSC) adopted Resolution 1696, calling for the suspension of all uranium enrichment and reprocessing related activities, until IAEA officials could assess the nature of Iran's enrichment activity.\cite{resolution1696}

While UN resolutions were aimed to set a ceiling for Iran's enrichment capabilities, analysis of IAEA inspection data revealed that Iran could further enrich its Low Enriched Uranium (LEU)\footnotemark[1] stockpiles.  Therefore, Iran was capable of constructing a nuclear weapon by the beginning of 2011.\cite{hasIranAchievedaNuclearWeapon}

\footnotetext[1]{Low Level Uranium (LUE) enrichment generates the fuel consumed in a civilian nuclear reactor, a process where naturally found uranium is refined to comprise between 7-20\% Uranium-235 as fissile material.\cite{uraniumEnrichmentDetails}}

In June 2010, a Belarusian malware-detection firm detected malware on client computers fraudulently signed by two different digital certificates, one from Realtek Semiconductor, and the other from JMicron Technology.\cite{theRealStoryOfStuxnet} This unprecedented virus's construction employed legitimate keyfile signatures, infecting Windows computers found to be concentrated in Iran.\cite{w32.stuxnetDossier} Security analysts extensively analyzed the virus, to determine its intent, and once identified, Microsoft coined the name Stuxnet to expose its threat to vulnerable Windows computer systems.\cite{microsoftCoinsStuxnet}

On November 3rd, 2010, the head of Iran's Atomic Energy Association reported nuclear enrichment capability losses after malware caused their facilities to suffer critical malfunctions the prior year.\cite{didStuxnetTakeOut1000Centrifuges} 

Stuxnet's code targeted a unique hardware configuration for a particular Supervisory Control and Data Acquisition System (SCADA), which monitored, and injected code into programmable logic controllers (PLCs). Before the viruses' target or intent were uncovered, Stuxnet was able to breach Iran's Nantez uranium enrichment facility, modify the frequencies at which enrichment centrifuge motors spun. Moreover, the virus manipulated the on-site systems monitoring results with unaltered system records. The result of Stuxnet's deployment was catastrophic physical destruction of Iranian critical nuclear infrastructure.\cite{w32.stuxnetDossier}\cite{lessonsFromStuxnet}

%%%%%%%%%%%%%%%%%%%%%%%%%
%%% Focus Question %%%
%%%%%%%%%%%%%%%%%%%%%%%%%
\section{Focus question}

Was it ethical for the United States and Israel to destroy a portion of Iran's critical nuclear infrastructure with the deployment of the Stuxnet virus?

%%%%%%%%%%%%%%%%%%%%%%%%%
%%% Social Implications %%%
%%%%%%%%%%%%%%%%%%%%%%%%%
\section{Social Implications}

\subsection{Iran Retaliates against the United States}
A Grand Jury in the Southern District of New York indicted seven Iranian citizens employed by Iran based companies acting on behalf of the Islamic Revolutionary Guard. The employees found responsible were charged with a campaign of Denial-of-Service (DDoS) attacks beginning in 2011 which, ”disabled victim bank websites, prevented customers from accessing their accounts online and collectively cost the victims tens of millions of dollars in remediation costs.”\cite{sevenIraniansIndicted} This indictment reveals how cyber attacks that interfere with critical infrastructure, in this case the US Banking system, have followed the unprecedented US attack against Iran.

Furthermore, in the indictment, Hamid Firoozi was charged with, ”obtaining unauthorized access into the Supervisory Control and Data Acquisition (SCADA) systems of the Bowman Dam, located in Rye, New York.”\cite{sevenIraniansIndicted} The United States Department of Justice noted, ”unauthorized access allowed him to obtain information regarding the status and operation of the dam [including:] water levels, temperature and status of the sluice gate, which is responsible for controlling water levels and flow rates.”\cite{sevenIraniansIndicted} Hydro-electric energy facilities are components of a country’s critical infrastructure. In defense of the indictment, Assistant Attorney General Carlin stated,

\begin{displayquote}
”Like past nation state-sponsored hackers, these defendants and their backers believed that they could attack our critical infrastructure without consequence, from behind a veil of cyber anonymity.”
\end{displayquote}

These attacks by the Iranian hackers are strikingly similar to the attack that the U.S and Israel launched against Iran.

\subsection{Constantly Increasing Number of Attack Vectors}

Today, the worldwide adoption of computer system based infrastructure creates an environment with an increasing number of attack vectors for hackers to discover and use to infiltrate. The number of disclosed security vulnerabilities increased 29.2\% in the first quarter of 2017, revealing an increase in opportunities to attack.\cite{industrialCyberVulnerabilities} These statistics, paired with Stuxnet’s precedent setting physical impact, indicate a likelihood more viruses will be developed targeting critical infrastructure for malicious reasons.[19] We can anticipate a rise in these types of attacks against countries that that are reliant on computer systems to control their critical infrastructure.

\subsection{Barack Obama on National Security}

In February 2013, President Barack Obama issued Presidential Policy Directive 21 requiring all federal contracts to, ”take proactive steps to manage risk and strengthen the security and resilience of the Nation’s critical infrastructure against both physical and cyber threats.”\cite{industrialCyberVulnerabilities} This presidential policy statement, made a couple of years following the Stuxnet attack on Iranian critical infrastructure, highlights the understood urgency of taking precautions against cyber threats leveled against nation-states.

%%%%%%%%%%%%%%%%%%%%%%%%%%%%%%%%%%%%%%%%%%%%%%
%%% Extant Arguments from External Sources %%%
%%%%%%%%%%%%%%%%%%%%%%%%%%%%%%%%%%%%%%%%%%%%%%
\section{Others' arguments}

\subsection{Ethical}

It is ethical for a state-sponsored virus to target and disrupt another state's critical infrastructure.

\subsubsection{Iran's nuclear program is being targeted for violating numerous UNSC Resolutions pertaining to uranium enrichment control}

The United States and Israel understood that Iran was nearing a state of nuclear "break-out". A state where Iran possesses the capability of enriching enough uranium, before IAEA inspections, or international intelligence believe they could confidently intercept enrichment activity. Experts postulate that as a preventative measure to promote nuclear non-proliferation, the United States and Israel jointly agreed to limit the efficiency of Iran's nuclear enrichment through the deployment of Stuxnet, resulting in the increased likelihood IAEA officials would intercept troubling enrichment activity associated with nuclear weapons development.\cite{theRealStoryOfStuxnet}

\subsubsection{Military intervention would likely result in civilian casualties in addition to environmental damage}

The United States and Israel drafted plans for military bombings of nuclear enrichment facilities. Reuter's cited policy statements on January 29, 2009, "the United States [reserves] all its options, ranging from diplomacy to military action, to pressure Iran over its nuclear program".\cite{usOptionsForIran} Even the most precise military intervention on nuclear infrastructure, there's a high probability of civilian casualties. In addition, such action would likely result in the release of radioactive particles into the atmosphere, creating an environmental crisis. This compounding consequence could result in additional civilian casualties. The careful design and deployment of the Stuxnet virus circumvented environmental damage and casualties, while protecting the interests of international nuclear non-proliferation.

\subsection{Unethical}

It is unethical for a state-sponsored virus to target and disrupt another state's critical infrastructure.

\subsubsection{Stuxnet targeted Iranian critical nuclear infrastructure, resulting in physical destruction}

From evidence gathered, the Stuxnet virus was responsible for the necessary replacement of an estimated 984 of the roughly 9000 total centrifuges at the Nantez nuclear enrichment facility, resulting in the disruption of Iran's Low Level Enrichment capabilities.\cite{lookIntoIranianNuclearProgram} This action disregarded international sovereignty laws, disrupting Iranian critical nuclear enrichment production, restricting the generation of fuel used in civilian nuclear reactors.\cite{internationalSovereigntyDefinition}\footnotemark[2] 

\footnotetext[2]{International sovereignty is well protected by the United Nation's charter and is further defined by the General Assembly in A/RES/50/172 to include: "the strict observance by States of the obligation not to intervene in the affairs of any other State is an essential condition to ensure that nations live together in peace with one another, since the practice of any form of intervention not only \underline{violates the spirit and letter of the Charter}, but also leads to the \underline{creation of situations which threaten international peace and security}\cite{internationalSovereigntyDefinition}}

\subsubsection{Stuxnet infected over 50,000 unique organizations' IP addresses}

Using fraudulent digital certificates, Stuxnet obscured itself as licensed software from two major companies Realtek Semiconductor and JMicron Technology. With these signatures, experts estimate the virus infected over 50,000 unique organizations' IP addresses.\cite{lessonsFromStuxnet} Records indicate a concentration of infected IP addresses in Iran, calling to question the ethics of victimizing private corporations, in addition to  utilizing them as channels of disrupting their respective nation's own critical infrastructure. Stuxnet infiltrated the computer systems of private individuals and organizations in Iran to construct a source of an attack vector, ultimately successful, subsequently breaching Iran's Nantez nuclear enrichment facility.\cite{w32.stuxnetDossier} 

\subsubsection{Stuxnet was not officially sanctioned by UNSC resolutions}

UNSC Resolution 1747 emphasizes,
\begin{displayquote}
"The importance of political and diplomatic efforts to find a negotiated solution guaranteeing that Iran’s nuclear program is exclusively for peaceful purposes ... noting that such a solution would benefit nuclear non-proliferation ... welcoming the [continued] commitment of ... the United States ... to seek a negotiated solution."\cite{resolution1747}
\end{displayquote}

The resolution was adopted to deter the likelihood a nation would physically intervene in order to deter Iran's pursuit of nuclear weapons. The Untied States violated key aspects of this resolution as Stuxnet resulted in the physical destruction of Iran's critical nuclear infrastructure, bypassing diplomatic negotiation.


%%%%%%%%%%%%%%%%%%%%%%%%%%%
%%% Analysis %%%
%%%%%%%%%%%%%%%%%%%%%%%%%%%
\section{Analysis}

\subsection{How the Software Engineering Code of Ethics applies}

In its preamble, the SE Code states it “prescribes these as obligations of anyone claiming to be or aspiring to be a software engineer.”\cite{softwareEngineeringCodeOfEthics}

\subsubsection{Software Engineer}

The SE Code states that Software Engineers are “those who contribute by \underline{direct participation} … to the … \underline{design} … of \underline{software systems}.”\cite{softwareEngineeringCodeOfEthics}

\paragraph{Direct Participation}

Direct means, “without an intervening agency.”\cite{merriamWebsterDefinitions} Participation means, “to take part in something.”\cite{cambridgeDictionary} An agency is, “a person or thing through which power is exerted or an end is achieved.”\cite{merriamWebsterDefinitions} Therefore, direct participation means taking part without an intervening person or thing through which power is exerted or an end is achieved.

\paragraph{Design}

Design is the, "planning that exists behind an action."\cite{oxfordDictionary}

\paragraph{Software System}

Software is defined as, ”the programs used to direct the operation of a computer”.\cite{softwareDefinition} A system is defined as, ”An organized or established procedure.”\cite{merriamWebsterDefinitions} Therefore, a software system can be defined as the programs used to direct the operation of a computer, comprised of an organized or established procedure. Stuxnet was a program with established procedures used to inject itself into other computer networks, in addition to procedures for interfering with the Nantez enrichment facility operations. Therefore, the Stuxnet virus is a software system.\cite{w32.stuxnetDossier}

\paragraph{Software Engineer Definition}

Using the above definitions, a Software Engineer is one who contributes by taking part without an intervening person or thing through which power is exerted or an end is achieved to devise, for a specific function or end, a weaponized computer virus.

\subsubsection{Analysis}

\paragraph{Contributes by Taking Part}

Stuxnet's software design was managed by covert state-sponsored programs under the leadership of both the United States and Israel. These programs were housed within the National Security Agency and Unit 8200 respectively, both of which held direct participation in the design and false certification of the virus's 50 kilobyte code base.\cite{NationalSecurityAgencyAndUnit8200}\cite{w32.stuxnetDossier} Therefore, both the United States and Israeli governments contributed by taking part in the development of Stuxnet.

\paragraph{Without an Intervening Person}

Stuxnet needed to breach the air-gap the Nantez nuclear facility innately exhibits due to its closed access from the internet. This being the case, Stuxnet was designed as a fire-and-forget type weapon, where once deployed, its execution could not be externally terminated.\cite{stuxnetFireandForget} The nature of this virus prevents outside involvement or interference by another party outside of its initial developers, until it can be identified and reverse engineered by security analysts. Stuxnet was deployed after the green-light of government officials in both the United States and Israel. Therefore, once deployed, the virus executed without an intervening party.

\paragraph{Devise for a Specific Function or End}

The Stuxnet virus was used to destroy a portion of Iranian enrichment infrastructure.\cite{theRealStoryOfStuxnet} Therefore, the Stuxnet virus was devised for the specific end of controlling nuclear weapons development.

\subsubsection{Conclusion}
Software Engineers sponsored by the United States and Israel were directly involved in the development of Stuxnet.

\subsubsection{Policymakers}

The SE Code of Ethics contains, “Principles related to the behavior of and decisions made by professional software engineers, including … policymakers.” Furthermore, the Code states, “The Principles identify the ethically responsible relationships in which individuals, groups, and organizations participate and the primary obligations within these relationships."\cite{softwareEngineeringCodeOfEthics}

\paragraph{Policymakers Definition}

Policymaker is defined, “A person responsible for or involved in formulating policies.”\cite{oxfordDictionary}

\subparagraph{Policy Definition}

Policy is defined, “Principle of action adopted or proposed by an organization.”\cite{oxfordDictionary}

\subparagraph{Formulate Definition}

Formulate means, “create or devise.”\cite{oxfordDictionary}

\subsubsection{Who the SE Code Applies to}

A domain specific version of who the SE Code applies to with these definitions becomes: persons responsible for or involved in creating or devising principles of action adopted or proposed by an organization.

\paragraph{Analysis}

As defined above, Stuxnet was developed by Software Engineers sponsored by the United States and Israel. Therefore, these state-sponsors acted as policymakers to direct the behavior of and decisions made by professional software engineers.

\paragraph{Conclusion}

As the United States and Israel acted as policymakers directing the actions of Software Engineers in the development of Stuxnet, they are responsible for adhering to the tenets outlined in the SE Code of Ethics. Therefore, we will analyze the ethics of the state-sponsored design of Stuxnet by applying the SE Code Tenets.

\subsection{Code Tenet 3.14}

\subsubsection{Definition}

Tenet 3.14 states:

\begin{framed}
Software Engineers shall, “\ul{maintain} the \ul{integrity} of \ul{data}, being \ul{sensitive} to \ul{outdated} … occurrences.”\cite{softwareEngineeringCodeOfEthics}
\end{framed}

\paragraph{Maintain}

Maintain is defined, “to keep in an existing state.”\cite{merriamWebsterDefinitions}

\paragraph{Integrity}

Integrity is defined, “lack of corruption in electronic data.”\cite{oxfordDictionary} Data was previously defined.

\paragraph{Data}

Data is defined, “facts and statistics collected together for reference or analysis.”\cite{dataDefinition}

\paragraph{Sensitive}

Sensitive means to be, “quick to detect or respond to … signals.”\cite{oxfordDictionary} Signals is defined, “An indication of a situation.”\cite{oxfordDictionary} Therefore being sensitive with respect to signals means being quick to detect or respond to indications of a situation.

\paragraph{Outdated}

Outdated means, “no longer current.”\cite{merriamWebsterDefinitions}

\subsubsection{Domain Specific Tenet}

Therefore, the SE Code tenet can be stated:

\begin{framed}
The state-sponsors of Stuxnet must \ul{keep in its existing state the lack of corruption of facts} and statistics \ul{collected together for reference or analysis}, being quick to \ul{detect indications of a situation resulting} from \ul{no longer current occurrences.}
\end{framed}

\subsubsection{Analysis}

\paragraph{Keep in its existing state the lack of corruption of facts}

After successfully breaching the Nantez enrichment facility, the Stuxnet virus collected data from properly operating centrifuges. Then, Stuxnet activated its payload and overwrote SCADA monitoring equipment with the previously recorded data.\cite{lessonsFromStuxnet} This false data corrupted the true factual operating statistics of enrichment centrifuge controllers.

\paragraph{Collected Together for Reference or Analysis}

In the case of the Nantez enrichment facility, Supervisory Control and Data Acquisition software interfaces with hardware centrifuges to operate properly.

\subparagraph{Supervisory Control and Data Acquisition (SCADA)}

Schneider Electric states, “SCADA Host platforms … provide functions for graphical displays, alarming, trending and historical storage of data,” all of which is to be consumed as accurate current operating conditions for Nantez facility operators.\cite{schneiderElectric}

With this evidence, we can conclude that the Nantez facility utilized SCADA systems to reference and analyze the current operating conditions of their enrichment centrifuges.


\paragraph{Detect Indications of a Situation Resulting}

The Stuxnet virus was designed to mask factual operating data, thus preventing operators from being able to detect indications of situational centrifuge failure. The United States and Israel intentionally made it impossible to monitor how the centrifuges were currently operating. They made no attempt to detect indications that would identify erroneous data. The purpose of the attack was to obscure accurate data.

\paragraph{No Longer Current Occurrences}

The Stuxnet virus recorded past operating conditions overwriting current data to conceal its attack vector.  Its mode of attack was to modify the speeds at which centrifuges spun. Centrifuge operators were not able to view current data which would have allowed them to diagnose their equipments’ critical condition. Instead, the compromised centrifuges exceeded design specifications and catastrophically malfunctioned.

\subsubsection{Conclusion}

The Stuxnet virus purposely compromised the integrity of factual operating informations of the SCADA Software Systems at the Nantez nuclear enrichment facility, in addition to concealing its effects. This design violates Tenet 3.14 of the Code of Ethics and is therefore unethical.


\subsection{Code Tenet 3.13}

\subsubsection{Definition}

Tenet 3.13  states:
\begin{framed}
Software Engineers shall, "Be careful to \ul{use only accurate data} \ul{derived} by \ul{ethical} and \ul{lawful} means, and \ul{use it only in ways properly authorized}."\cite{softwareEngineeringCodeOfEthics}
\end{framed}

\paragraph{Use only accurate data}
Accurate is defined as, "being in agreement with the truth," while data is defined as, "facts and statistics collected together for reference or analysis."\cite{cambridgeDictionary} Therefore, using only accurate data means to only use facts and statistics collected together for reference or analysis that are in agreement with the truth. The Stuxnet virus relied on presenting inaccurate data to execute and conceal its intended operations.

\paragraph{Derived}
The definition of derived is, "to receive or obtain from a source or origin."\cite{softwareDefinition}

\subparagraph{Derived ethically}
Ethical is defined as, "conforming to accepted standards of conduct."\cite{cambridgeDictionary} With this consideration, something is derived ethically when it's produced by actions that conform to the accepted standards of conduct.

\subparagraph{Derived lawfully}
Lawful means, "being in harmony with the law," therefore, an action that is performed lawfully, adheres to the all laws and statutes adopted by the accepted governing legislation.\cite{cambridgeDictionary} With this consideration, something is derived lawfully when it is produced by actions that adhere to all laws and statutes respectively governing the action.

\paragraph{Only use data when properly authorized}

Authorized software is typically identified by the operating system provider.  The client relies on Microsoft to provide this software authorization.  About Stuxnet, Windows stated in their security bulletin:

\begin{displayquote}
“By placing Stuxnet on the Severe alert level of their security bulletin and indicating that “it provides the capability for hackers to obtain backdoor accessibility to your PC”.\cite{w32.stuxnetDossier}
\end{displayquote}

Windows through their monitoring system, indicated that Stuxnet was not authorized software. Since Stuxnet was not authorized, it should not have been propagated by the individuals operating on behalf of the United States and Israel.


\subsubsection{Domain Specific Tenet}

Therefore, SE Code Tenet 3.13 can be written:

\begin{framed}
Participants in the development of Stuxnet should be careful to \ul{only use facts and statistics collected for reference that are in agreement with the truth} in addition to \ul{using this information only when properly authorized by appropriate systems operators} at the Nantez facility.
\end{framed}

\subsubsection{Analysis}

\paragraph{Only use facts and statistics that are in agreement with the truth}

See previous argument.

\subparagraph{Data Collection and Presentation}

After successfully breaching the Nantez enrichment facility, the Stuxnet virus collected data from properly operating centrifuges, therefore its data collection was in agreement with the truth. However, after secretly monitoring enrichment activity for a period of time, Stuxnet activated its payload and overwrote SCADA monitoring equipment with the previously recorded data.\cite{toKillACentrifuge} This action presented on-site operators at the Nantez enrichment facility false data, purposely collected and applied to be in disagreement with the truth, therefore it is in violation the SE Code of Ethics.

\paragraph{Properly Authorized by Appropriate Systems Operators}

Bypassing Nantez facility authorization, on deployment of Stuxnet's payload, previously recorded accurate monitoring metrics were used to overwrite the true SCADA System statistics, concealing modified operating conditions.\cite{theRealStoryOfStuxnet} The virus's code, or information, executed unknowingly, and therefore without proper authorization from officials monitoring Nantez facility operations.

In addition, the Stuxnet virus bypassed Windows security procedures with fraudulent digital certificates, therefore it executed its code without proper authorization from the Windows OS. Furthermore, as the virus utilized the digital certificate, or information fraudulently, it did so without any authorization from either JMicron or RealTek.\cite{signedUsingCertificates} This is in direct violation of using information only when properly authorized.

\subparagraph{Fraudulent Digital Certification}

The Windows Development Center begins its documentation on Digital Certificates with a warning. They state that due to the nature of computer-to-computer communication, it exposes an opportunity that could, "allow an \ul{unethical} person to intercept messages or to impersonate another person or entity."\cite{moreOnDigitalCertificates}

In order for Stuxnet to deliver its payload to the Nantez enrichment facility, it built a network of potential attack vectors by infecting thousands of computers largely centralized in Iran.\cite{lessonsFromStuxnet} The virus was successful in executing due to the inclusion of fraudulent digital certificates to circumvent Windows system security measures.

Stuxnet's Software Engineers acquired stolen digital certificates from two Taiwanese corporations, Realtek and JMicron.\cite{signedUsingCertificates}. Due to the nature of the certificates representing the two victim corporations by name, the Software Engineering teams responsible had first-hand knowledge Stuxnet was fraudulently signed. Therefore, these teams acted unethically as they presented digital certification information as factual that was not in agreement with the truth obscuring the malicious code to bypass Windows security procedures. The fraudulent digital certification of Stuxnet is in direct in violation of tenet 3.13.

\subsubsection{Conclusion}

The virus’s use of fraudulent digital certificates to infiltrate Nantez facility Windows Systems, bypasses proper authorization, and is therefore unethical when apply SE Code Tenet 3.13.

\subsection{Code Tenet 1.01}

\subsubsection{Definition}

Tenet 1.01 states:
\begin{framed}
Software Engineers shall, "\ul{Accept full responsibility for their own work}."\cite{softwareEngineeringCodeOfEthics}
\end{framed}

\paragraph{Accept full responsibility for their own work}

Accept is defined as, "to recognize as true."\cite{merriamWebsterDefinitions} Responsibility is defined as, "moral, legal, or mental accountability."\cite{merriamWebsterDefinitions}

Work is defined as an, "activity in which one exerts strength or faculties to do or perform something."\cite{cambridgeDictionary} Previously defined, in this domain, the actor's own work is the Stuxnet virus.

Therefore, one accepts full responsibility for their own work when they identify in totality  the, moral, legal, and mental repercussions consequent to their actions, in addition to recognizing such consequences as the truth.\cite{cambridgeDictionary}

\subsubsection{Domain Specific Code Tenet}

With these definitions, the SE Code Tenet 1.01 can be interpreted:
\begin{framed}
The United States and Israel should \ul{identify in totality}, the \ul{moral}, \ul{legal}, and \ul{mental} repercussions resultant from their work, in addition to \ul{recognizing such consequences as the truth}.
\end{framed}

\subsubsection{Analysis}

\paragraph{Identify in totality the repercussions}

\subparagraph{Moral}

In Sovereignty and Morality, speaking on the foundations of Sovereignty founded in moral obligation, W. E. Hocking writes,

\begin{displayquote}
"If I refrain on moral grounds from breaking the furniture of a person whose taste in furniture I disapprove, it is not primarily because I have set over myself a power which renders it imprudent to act in this way.; it is rather because he and I alike prefer to live in a world where freedom of choice ... is respected, and because he and I are capable of recognizing in one another that preference ... as a common good and a common bond."\cite{soverigntyAndMoralObligation}
\end{displayquote}

This poetic representation reveals how the adoption of Internationally defined Sovereignty is in pursuit for common good and mutual cooperation. The deployment of Stuxnet disregarded existing United Nations Resolutions, going against international cooperative efforts.\cite{resolution1747} This action is morally questionable as it violated Iran's sovereignty.

\subparagraph{Legal}

In Science and Trans-Science, Alvin M. Weinburg writes,

\begin{displayquote}
"The politician, or some other representative of society, is ... expected to say whether the society ought to proceed in one direction or another. The scientist and science provide the means; the politician and politics decide the ends."\cite{scienceAndTransScience}
\end{displayquote}

When an action is never publicly recognized by the responsible party, consequently there remains no opportunity to express opinions regarding the consequences of an action, with respect to the actor. Therefore, the totality of the legal repercussions resultant from Stuxnet's deployment were never fully conceived, as neither the United States' nor Israeli government publicly recognized their involvement.

\subparagraph{Mental}

Advocacy groups such as the American Civil Liberties Union and the Electronic Frontier Foundation have taken a part in sharing leaked classified United States' documentation following the deployment of Stuxnet. Documents included a discussion that Iran's recent cyber attacks were, 

\begin{displayquote}
"in retaliation to Western activities against Iran's nuclear sector and that senior officials in the Iranian government are aware of these attacks. NSA expects Iran will continue this series of attacks, which it views as successful, while striving for increased effectiveness by adapting its tactics and techniques to circumvent victim mitigation attempts."\cite{effForwardsIranDiscussion}\cite{acluForwardsIranDiscussion}
\end{displayquote}

This reveals a portion of the mental repercussions at play, as it shows the United States recognized their influence in Iran's decisions to retaliate with similar cyber attacks.  

\paragraph{Recognize the resulting consequences as the truth}

Neither the United States, nor Israel have officially publicly announced their involvement in the development of Stuxnet, likely for the purposes of military concealment. Such position has restricted the international community's ability to review Stuxnet's ethically questionable nature. The result has been precedent setting minimal repercussions for what has been recognized as the first use of state-sanctioned cyber-weapon. By never publicly admitting involvement in the development of Stuxnet, they never recognized the resulting consequences of their deployment of Stuxnet as the truth. Therefore, both the United States and Israel acted unethically by not disclosing a precedent setting act of international cyber intervention.

\subsubsection{Conclusion}

By never publicly admitting involvement in the development of Stuxnet, both the United States and Israel acted unethically. As a result, the legislative and judicial departments of the United States could not discuss the consequences of Stuxnet, thus the moral, legal, and mental repercussions were never fully conceived or thoroughly discussed in public forums. Therefore, the actions taken after the deployment of Stuxnet are unethical when applying SE Code Tenet 1.01.


%%%%%%%%%%%%%%%%%%%%%%%%%%%%%%%%%%%%%%%
%%% Conclusion %%%
%%%%%%%%%%%%%%%%%%%%%%%%%%%%%%%%%%%%%%%
\section{Conclusion}

In conclusion, both the United States and Israel acted unethically when they held direct-participation in the development and deployment of the Stuxnet virus. Their actions violated international sovereignty law, restricting Iran's ability to develop its nuclear program. Iran's strict adherence to the IAEA's inspection regime at the time of Stuxnet's initial deployment further calls to question the legitimacy of the United State's and Israel's ardent stance of deterring Iran from developing nuclear infrastructure all-together.

At a low-level, Stuxnet operated by forging critical nuclear infrastructure monitoring equipment data readouts, interfering with the operations of on-site workers, resulting in civilians losing their employment. This action violates the SE Code Tenet 3.13 as it potentially violates the United States' own definition of espionage, but more certainly is classified as the unauthorized access and use of information.

Furthermore, the public statements following the deployment of Stuxnet by the United States and Israel are found to be unethical when applying SE Code Tenet 1.01 as their silence prevented the totality the moral, legal, and ethical repercussions from being discussed in public forums.

Stuxnet's intended operation was successful in destroying 984 enrichment centrifuges at the Nantez facility, hindering Iran's enrichment potential. While Stuxnet's immediate effects were only temporary,  its precedent established for with regards to state-sponsored cyber-weapons could have dramatic, long lasting impact in the future.


\end{multicols}

%cite all the references from the bibtex you haven't explicitly cited
%\nocite{*}

\bibliographystyle{IEEEannot}

\newpage
\bibliography{bibliography}
\end{document}
