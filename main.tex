% Term paper proposal - Andrew Exton
% CSC 300: Professional Responsibilities
% Dr. Clark Turner

% One Column Format
\documentclass[12pt]{article}

\usepackage{setspace}
\usepackage{url}
\usepackage{multicol}
\usepackage{footmisc}
\usepackage{titlesec}
\usepackage{framed}
\usepackage{csquotes}
\usepackage{soul}


%%% INDENTATION DEPTH
\setcounter{secnumdepth}{5}
\setcounter{tocdepth}{5}

%%% PARAGRAPH TITLE FORMAT
\titleformat{\paragraph}
{\normalfont\normalsize\bfseries}{\theparagraph}{1em}{}
\titlespacing*{\paragraph}
{0pt}{3.25ex plus 1ex minus .2ex}{1.5ex plus .2ex}

%%% SUBPARAGRAPH TITLE FORMAT
\titleformat{\subparagraph}
{\normalfont\normalsize\bfseries}{\thesubparagraph}{1em}{}
\titlespacing*{\subparagraph}
{0pt}{3.25ex plus 1ex minus .2ex}{1.5ex plus .2ex}

%%% PAGE DIMENSIONS
\usepackage{geometry} % to change the page dimensions
\geometry{letterpaper}


\begin{document}

\title{\vfill Stalling Iranian Uranium Enrichment: a Stuxnet Case Study } %\vfill gives us the black space at the top of the page
\author{
By Andrew Exton \vspace{10pt} \\
CPE 300: Professional Responsibilities  \vspace{10pt} \\
Dr. Clark Turner \vspace{10pt} \\
}
\date{\today} %Or use \today for today's Date

\maketitle

\vfill  %in combination with \newpage this forces the abstract to the bottom of the page
\begin{abstract}
Throughout the early 2000s, Iran's policy on nuclear development, in addition to international intelligence gathering indicated Iran sought the advancement of its nuclear weapons program, grounding concern and sparking debate in forums such as the United Nations.\cite{unitedNationsResolutions}
Was it ethical for the United States and Israel to intervene, controlling Iran's uranium production efficiency by deploying the Stuxnet virus, destroying critical nuclear infrastructure?

Some argue it limited Iran's ability to persue nuclear weapons development.\cite{theRealStoryOfStuxnet} Others suggest this deliberate action restricted Iran's ability to pursue nuclear energy for civilian infrastructure.\cite{lookIntoIranianNuclearProgram} By applying tenets 3.13 and 1.01 of the Software Engineering Code of Ethics, we determine that this use of a state-sponsored cyber-weapon is unethical because it violates Iran's sovereignty.

%%% MAIN CODE TENET + PLAIN LANGUAGE REASON !!!!!

\end{abstract}

\thispagestyle{empty} %remove page number from title page, but still keep it as pg #1
\newpage

\tableofcontents
\newpage

\begin{multicols}{2}

%%%%%%%%%%%%%%%%%%%%
%%% Known Facts  %%%
%%%%%%%%%%%%%%%%%%%%
\section{Facts}

Throughout the early 2000s, the Islamic Republic of Iran's nuclear program was a topic of grave concern and sparked debate in forums such as the United Nations (UN).\cite{unitedNationsResolutions} The UN imposed international mandates on the Iran, requiring the disclosure of Iran's nuclear program activity to the International Atomic Energy Association (IAEA), following state policy statements targeting their nuclear weapon program advancements. 

After reviewing Iran's October 21st, 2003 Nuclear Program Declaration, the IAEA questioned why certain centrifuge designs uncovered during the inspection were not included in the report, which Iranian authorities responded they, "neglected to include due to time considerations".\cite{implementationOfNPTSafeguards}

In response to Iran's continued enrichment actions, which violated previous UN Resolutions, the United Nations Security Council (UNSC) adopted Resolution 1696, calling for the suspension of all uranium enrichment and reprocessing related activities, until IAEA officials could assess the nature of Iran's enrichment activity.\cite{resolution1696}

While UN resolutions were aimed to set a ceiling for Iran's enrichment capabilities, analysis of IAEA inspection data revealed that Iran could further enrich its Low Enriched Uranium (LEU)\footnotemark[1] stockpiles.  Therefore, Iran was capable of constructing a nuclear weapon by the beginning of 2011.\cite{hasIranAchievedaNuclearWeapon}

\footnotetext[1]{Low Level Uranium (LUE) enrichment generates the fuel consumed in a civilian nuclear reactor, a process where naturally found uranium is refined to comprise between 7-20\% Uranium-235 as fissile material.\cite{uraniumEnrichmentDetails}}

In June 2010, a Belarusian malware-detection firm detected malware on client computers fraudulently signed by two different digital certificates, one from Realtek Semiconductor, and the other from JMicron Technology.\cite{theRealStoryOfStuxnet} This unprecedented virus's construction employed legitimate keyfile signatures, infecting Windows computers identified by private organization IP addresses, found to be concentrated in Iran.\cite{w32.stuxnetDossier} Security analysts extensively analyzed the virus, to determine its intent, and once identified, Microsoft coined the name Stuxnet to expose its threat to vulnerable Windows computer systems.\cite{microsoftCoinsStuxnet}

On November 3rd, 2010, the head of Iran's Atomic Energy Association reported nuclear enrichment capability losses after malware caused their facilities to suffer critical malfunctions the prior year.\cite{didStuxnetTakeOut1000Centrifuges} 

Stuxnet's code targeted a unique hardware configuration for a particular Supervisory Control and Data Acquisition System (SCADA), which monitored, and injected code into programmable logic controllers (PLCs). Before the viruses target or intent were uncovered, Stuxnet was able to breach Iran's Nantez uranium enrichment facility, modify the frequencies at which enrichment centrifuge motors spun, in addition to manipulating the on-site systems monitoring results with unaltered system records. The result of Stuxnet's deployment was catastrophic physical destruction of Iranian critical nuclear infrastructure.\cite{w32.stuxnetDossier}\cite{lessonsFromStuxnet}

%%%%%%%%%%%%%%%%%%%%%%%%%
%%% Research Question %%%
%%%%%%%%%%%%%%%%%%%%%%%%%
\section{Research question}

Was it ethical for the United States and Israel to destroy a portion of Iran's critical nuclear infrastructure with the deployment of the Stuxnet virus?

%%%%%%%%%%%%%%%%%%%%%%%%%
%%% Social Implications %%%
%%%%%%%%%%%%%%%%%%%%%%%%%
\section{Social Implications}

\subsection{Barack Obama on National Security}

In February 2013, President Barack Obama issued PPD-21 requiring all federal contracts to, "take proactive steps to manage risk and strengthen the security and resilience of the Nation’s critical infrastructure," cementing United State's policy to, "strengthen the security and resilience of its critical infrastructure against both physical and cyber threats."\cite{industrialCyberVulnerabilities} This presidential policy statement, made a couple of years following Stuxnet's consequent physical destruction, highlights the urgency of taking precautions against cyber threats.

\subsection{Equifax highlights Software System Incompetence}

It's is dependent on system administrators updating to the latest security patches and implementing the current best practices of secure software design to combat the future influx of malware. Unfortunately, recent events such as the Equifax leak highlight the incompetence of many regarding software security.\cite{equifaxHack}

\subsection{Constantly Increasing Number of Attack Vectors}

Today, the worldwide adoption of computer system based infrastructure creates an environment with an increasing number of attack vectors for hackers to discover and use to infiltrate. The number of disclosed security vulnerabilities increased 29.2\% in the first quarter of 2017, revealing an increase in opportunities to attack. These statistics, paired with Stuxnet's precedent physical impact, indicate a likelihood more viruses are developed to target critical infrastructure for malicious reasons.\cite{industrialCyberVulnerabilities}

%%%%%%%%%%%%%%%%%%%%%%%%%%%%%%%%%%%%%%%%%%%%%%
%%% Extant Arguments from External Sources %%%
%%%%%%%%%%%%%%%%%%%%%%%%%%%%%%%%%%%%%%%%%%%%%%
\section{Others' arguments}

\subsection{Ethical}

It is ethical for a state-sponsored virus to target and disrupt another state's critical infrastructure.

\subsubsection{Iran's nuclear program is being targeted for violating numerous UNSC Resolutions pertaining to uranium enrichment control}

The United States and Israel understood that Iran was nearing a state of nuclear "break-out". A state where Iran possesses the capability of enriching enough uranium before IAEA inspections, or international intelligence believes they could confidently intercept enrichment activity. Experts postulate that as a preventative measure to promote nuclear non-proliferation, the United States and Israel jointly agreed to limit the efficiency of Iran's nuclear enrichment through the deployment of Stuxnet, resulting in the increased likelihood IAEA officials would intercept troubling enrichment activity associated with nuclear weapons development.\cite{theRealStoryOfStuxnet}

\subsubsection{Military intervention would have resulted in environmental damage, in addition to civilian casualties}

The United States and Israel drafted plans for military bombings of nuclear enrichment facilities. Reuter's citing policy statements on January 29, 2009, "the United States [reserves] all its options, ranging from diplomacy to military action, to pressure Iran over its nuclear program".\cite{usOptionsForIran} Even the most precise military intervention on nuclear infrastructure, would likely resulting in the release of radioactive particles into the atmosphere, creating an environmental crisis, subsequently resulting in civilian casualties. The careful design and deployment of the Stuxnet virus circumvented environmental damage and casualties, while protecting the interests of international nuclear non-proliferation.

\subsection{Unethical}

It is unethical for a state-sponsored virus to target and disrupt another state's critical infrastructure.

\subsubsection{Stuxnet targeted Iranian critical nuclear infrastructure, resulting in physical destruction}

From evidence gathered, the Stuxnet virus was responsible for the necessary replacement of an estimated 984 of the roughly 9000 total centrifuges at the Nantez nuclear enrichment facility, resulting in the disruption Iran's Low Level Enrichment capabilities.\cite{lookIntoIranianNuclearProgram} This action disregarded international sovereignty laws, disrupting Iranian critical nuclear enrichment production, allocated for use in civilian nuclear energy reactors.\cite{internationalSovereigntyDefinition}\footnotemark[2] 

\footnotetext[2]{International sovereignty is well protected by the United Nation's charter and is further defined by the General Assembly in A/RES/50/172 to include: "the strict observance by States of the obligation not to intervene in the affairs of any other State is an essential condition to ensure that nations live together in peace with one another, since the practice of any form of intervention not only \underline{violates the spirit and letter of the Charter}, but also leads to the \underline{creation of situations which threaten international peace and security}\cite{internationalSovereigntyDefinition}}

\subsubsection{Stuxnet infected over 50,000 unique organizations' IP addresses during its attack}

Using illegally procured certificate key-files, Stuxnet obscured itself as licensed software from two major companies Realtek Semiconductor and JMicron Technology. With these signatures, experts estimate the virus infected over 50,000 unique organizations' IP addresses.\cite{lessonsFromStuxnet} Records indicate a concentration of infected IP addresses in Iran, calling to question the ethics of victimizing private corporations, in addition to  utilizing them as channels of disrupting their respective nation's own critical infrastructure. Stuxnet infiltrated the computer systems of private individuals and organizations in Iran to construct a source of an attack vector, ultimately successful, subsequently breaching Iran's Nantez nuclear enrichment facility.\cite{w32.stuxnetDossier} 

\subsubsection{Stuxnet was not officially sanctioned by UNSC resolutions}

UNSC Resolution 1747 emphasizes,
\begin{displayquote}
"The importance of political and diplomatic efforts to find a negotiated solution guaranteeing that Iran’s nuclear program is exclusively for peaceful purposes ... noting that such a solution would benefit nuclear non-proliferation ... welcoming the [continued] commitment of ... the United States ... to seek a negotiated solution."\cite{resolution1747}
\end{displayquote}

The resolution was adopted to deter the likelihood a nation would physically intervene in order to deter Iran's pursuit of nuclear weapons. The Untied States violated key aspects of this resolution, as well as others, as it resulted in the physical destruction of Iran's critical nuclear infrastructure, bypassing diplomatic negotiation.


%%%%%%%%%%%%%%%%%%%%%%%%%%%
%%% Analysis %%%
%%%%%%%%%%%%%%%%%%%%%%%%%%%
\section{Analysis}

\subsection{How the Software Engineering Code of Ethics applies}

The preamble of the Software Engineering Code of Ethics extends its obligations to, "anyone claiming to be or aspiring to be a software engineer."\cite{softwareEngineeringCodeOfEthics}

\subsubsection{Software Engineer}

The SE Code further defines Software Engineers as, "those who contribute by \underline{direct participation} or by teaching, to the analysis, specification, \underline{design}, development, \underline{certification}, maintenance and testing of \underline{software systems}."

\paragraph{Direct Participation}

Combining the Cambridge Dictionary definitions for both Direct and Participation yields, "to take part ... in something without anyone or anything else being involved."\cite{cambridgeDictionary}

\paragraph{Design}
Design is the, "planning that exists behind an action."\cite{designDefinition}

\paragraph{Certification}
Certification is the act of certifying, respectively defined, "to attest as being true or as represented or as meeting a standard."\cite{cambridgeDictionary}

\paragraph{Software System}

Software Systems are defined as, "programs ... and associated documentation pertaining to the operation of a computer system."\cite{softwareSystemDefinition} Stuxnet was a program designed to inject itself into other computer networks, therefore in this domain the software system is the virus itself.\cite{w32.stuxnetDossier}

\subparagraph{Digital Certificates}
Pertaining to software systems, digital certificates are, "a set of data that completely identifies an entity, and [are] issued by a certification authority only after that authority has verified the entity's identity."\cite{digitalCertificateDefintion} These certificates sign software, marking the identity of the software's origins, which enables operating systems to authorize likely non-malicious executables to run. Therefore, the digital certificates used to sign the Stuxnet virus are encompassed by the definition for a software system.

\paragraph{Software Engineer Definition}

Viewed from a lens of domain specific applicability, the above definitions yield - a Software Engineer is one who contributes to the development of Stuxnet, independently from outside influence, in conjunction with any truthful testament that the respective programs and digital certification meet ethical standards.

\paragraph{Software Engineering Managers and Leaders Definition}

The SE Code of Ethics states, "Software engineering managers and leaders shall subscribe to and promote an ethical approach to the management of software development," further prescribing that they shall, "not ask a software engineer to do anything inconsistent with this Code."\cite{softwareEngineeringCodeOfEthics} Therefore, the actors that managed or led the development of the Stuxnet virus also are required to abide by, and govern their subordinates by, the ethical standards prescribed in the Code.


\subsection{Actor Analysis}

\subsubsection{Contributes by Taking Part}

Stuxnet's software design was managed by covert state-sponsored programs under the leadership of both the United States and Israel. These programs were housed within the National Security Agency and Unit 8200 respectively, both of which held Direct Participation in the design and false certification of the virus's 50 kilobyte code base.\cite{NationalSecurityAgencyAndUnit8200}\cite{w32.stuxnetDossier} Therefore, both the United States and Israeli governments contributed by taking part in the development of Stuxnet.

\subsubsection{Without Outside Involvement}

Stuxnet needed to breach the air-gap the Nantez nuclear facility innately exhibits due to its closed access from the internet. This being the case, Stuxnet was designed as a fire-and-forget type weapon, where once it was deployed its execution could not be externally terminated. The nature of the virus prevents outside involvement or interference by another party outside of its initial developers, until it can be identified and reverse engineered by security analysts.\cite{stuxnetFireandForget}


\subsubsection{Analysis}
Due to the Software Engineering Code of Ethics extension to the management and leadership of Software Engineering teams, it extends and is applicable to any nation-state allocating resources towards a development project - especially when funding is sourced from public taxation. The United States and Israel acted to fund Software Engineering teams in order to develop what ultimately became known as Stuxnet. The SE Code of Ethics describes obligations nation-states are required to adhere to when their objectives require Software Engineering.

\subsubsection{Conclusion}

The Software Engineering Code of Ethics documents an international consensus of the ethics practicing software engineers agree should be universally followed, "to ensure their efforts are used for good."\cite{softwareEngineeringCodeOfEthics} The United States and Israeli governments acted as leadership, to allocate resources for the development of Stuxnet. Therefore, both nations, contributed by taking part in the development of Stuxnet. Furthermore, their actions were performed without outside involvement as they act as sovereign nations. Therefore, both the United States and Israel are responsible for practicing ethical software development procedures when developing state sponsored software, such as Stuxnet. We will apply the SE Code of Ethics to analyze the ethics of state-sponsors pursuing cyberweapon development, utilizing Stuxnet as our case-study.


\subsection{Code Tenet 3.13}

\subsubsection{Definition}

Tenet 3.13  states:
\begin{framed}
Software Engineers shall, "Be careful to use only \ul{accurate data} derived by \ul{ethical} and \ul{lawful} means, and use it only in ways \ul{properly authorized}."\cite{softwareEngineeringCodeOfEthics}
\end{framed}

\paragraph{Use accurate data}
Accurate is defined as, "being in agreement with the truth," while data is defined as, "facts and statistics collected together for reference or analysis."\cite{cambridgeDictionary}

\paragraph{Derived ethically}
Ethical is defined as, "conforming to accepted standards of conduct."\cite{cambridgeDictionary} The Software Engineering Code of Ethics is an internationally accepted standard of conduct all Software Engineers should adopt.

\paragraph{Derived lawfully}
Lawful means, "being in harmony with the law," therefore, an action that is performed lawfully, adheres to the all laws and statutes adopted by the accepted governing legislation.\cite{cambridgeDictionary}

\paragraph{Only use data when properly authorized}
An action is properly authorized when it is performed after official approval, where the respective official has been designated to coordinate actions performed by its governing parties.

\subsubsection{Domain Specific Tenet}

Therefore, SE Code Tenet 3.13 can be written:
\begin{framed}
Software engineers should be careful to \ul{only use facts and statistics collected for reference that are in agreement with the truth}, \ul{derived by actions conforming to accepted standards of conduct}, and \ul{adhering to all laws and statutes governing}, in addition to \ul{using this information only when properly authorized by sanctioned officials}.
\end{framed}

\subsubsection{Analysis}

\paragraph{Only use facts and statistics that are in agreement with the truth}

\subparagraph{Data Collection and Presentation}

After successfully breaching the Nantez enrichment facility, the Stuxnet virus collected data from properly operating centrifuges, therefore its data collection was in agreement with the truth. However, after secretly monitoring enrichment activity for a period of 13 days, Stuxnet activated its payload and overwrote SCADA monitoring equipment with the previously recorded data. This action violates the SE Code of Ethics as it presented on-site operators at the Nantez enrichment facility false data, purposely collected and applied to be in disagreement with the truth.

\subparagraph{Fraudulent Digital Certification}

In order for Stuxnet to deliver its payload to the Nantez enrichment facility, it built a network of potential attack vectors by infecting thousands of computers largely centralized in Iran.

The false signing of the Stuxnet virus required the inclusion of digital signing certificates to circumvent Windows system security measures. These certificates constitute additional operating information needed by the virus, and are paired with the executable upon release.

Stuxnet's Software Engineers acquired stolen digital certificates from two Taiwanese corporations, Realtek and JMicron.\cite{signedUsingCertificates}.Due to the nature of the certificates specifically signed and represented the two victim corporations by name, the Software Engineering teams responsible had first hand knowledge Stuxnet was fradulently signed. Therefore, these teams acted unethically as they presented digital certification information as factual that was not in agreement with the truth in order to bypass Windows security systems. 

\paragraph{Derived by actions conforming to accepted standards of conduct}

Stuxnet recorded the normal operating conditions and metrics of the SCADA System at the Nantez enrichment facility silently, before deploying its payload. These metrics, while technically collected without authorization, never reached public networks, and thus does not constitute the majority of damage.

\paragraph{Adhering to all laws and statutes governing}

The means by which the virus breached the Nantez nuclear enrichment facility to perform data collection can be interpreted as espionage.

\subparagraph{Espionage}

The United States legal code defines espionage as,

\begin{displayquote}
"Gathering or delivering defense information," for the purpose of, "injury of the [victim nation] or to the advantage of a [another] nation,... \ul{transmits}... to any foreign government, or to any \ul{faction or party}... within a foreign country, whether recognized or unrecognized by the United States, or to any... \ul{employee, subject, or citizen thereof}, \ul{either directly or indirectly}, any \ul{document}, ... note, \ul{instrument}... shall be punished by death or by imprisonment for any term of years or for life."\cite{USEspionageLegalDefinition}
\end{displayquote}

With the advent of nuclear weapons, nuclear enrichment infrastructure has been intertwined with national defense. The United State's legal definition of espionage highlights how the Stuxnet virus could fall under the category of espionage, as the virus was transmitted with the purpose of injury to Iran's nuclear infrastructure.

Stuxnet infected thousands of IP addresses owned by private corporation in Iran. The virus code base (document) was then, unknowingly (indirectly), physically transfered by at least one employee, subject, or citizen working at the Nantez enrichment facility. The successful infiltration resulted in physical destruction of Iran's enrichment capabilities and thus national injury.

\paragraph{Using this information only when properly authorized}

Bypassing Nantez facility authorization, on deployment of Stuxnet's payload, previously recorded accurate monitoring metrics were used to overwrite the true SCADA System statistics, to conceal modified operating conditions.

\subparagraph{Hamid Firoozi Charged with Unauthorized Access into Bowman Dam SCADA System}

On March 24th, 2016 a grand jury in the Southern District of New York indicted seven Iranian citizens employed by Iran-based companies: ITSecTeam and Mersad Company, acting by the request of the Islamic Revolutionary Guard Corp. They were charged with a campaign of DDoS attacks which, "disabled victim bank websites, prevented customers from accessing their accounts online and collectively cost the victims tens of millions of dollars in remediation costs."\cite{sevenIraniansIndicted} In the same indictment, Hamid Firoozi was charged with, "obtaining unauthorized access into the Supervisory Control and Data Acquisition (SCADA) systems of the Bowman Dam, located in Rye, New York."\cite{sevenIraniansIndicted}

This United States Department of Justice noted, "unauthorized access allowed him to obtain information regarding the status and operation of the dam [including:] water levels, temperature and status of the sluice gate, which is responsible for controlling water levels and flow rates."\cite{sevenIraniansIndicted}

In defense of the indictment, Assistant Attorney General Carlin stated,

\begin{displayquote}
"Like past nation state-sponsored hackers, these defendants and their backers believed that they could attack our critical infrastructure without consequence, from behind a veil of cyber anonymity ... This indictment once again shows there is no such veil – we can and will expose malicious cyber hackers engaging in unlawful acts that threaten our public safety and national security.”\cite{sevenIraniansIndicted}
\end{displayquote}

\subsubsection{Conclusion}

The Stuxnet virus collected data from properly operating centrifuges at the Nantez nuclear enrichment facility in Iran. The United States and Israel acted unethically in both the collection of data and malicious use of respective data to conceal system problems, resulting in catastrophic malfunction. The data collection techniques employed by Stuxnet could constitute espionage under the United States legal definitions. In addition, the United States' Department of Justice indictment of Hamid Firoozi reveals unauthorized access to the SCADA Systems of critical infrastructure is a chargeable offense. The deployment of Stuxnet by the United States and Israel was therefore unethical.


\subsection{Code Tenet 1.01}

\subsubsection{Definition}

Tenet 1.01 states:
\begin{framed}
Software Engineers shall, "\underline{Accept full responsibility} for their \underline{own work}."\cite{softwareEngineeringCodeOfEthics}
\end{framed}

\paragraph{Accept full responsibility}
One accepts full responsibility when they identifies all the, "moral, legal, and mental" repercussions resultant from performing an action, while recognizing such consequences as true.\cite{cambridgeDictionary}\cite{cambridgeDictionary}

\paragraph{For own work}
Work is defined as an, "activity in which one exerts strength or faculties to do or perform something."\cite{cambridgeDictionary}

\subsubsection{Domain Specific Code Tenet}

Therefore, the SE Code Tenet 1.01 can be interpreted:
\begin{framed}
The United States and Israel should \ul{identify all the moral, legal, and mental repercussions} \ul{resultant from their work}, and \ul{recognize their consequences as the truth}.
\end{framed}

\subsubsection{Analysis}

\paragraph{Identify all Moral, Legal, and Mental Repercussions}

Violating the sovereignty for one nation to govern itself results in correlative repercussions.

\paragraph{Resultant from their work}

\paragraph{Recognize the resulting consequences as the truth}

\subsubsection{Conclusion}

Neither the United States, nor Israel have officially publicly announced their involvement in the development of Stuxnet, likely for the purposes of military concealment. Such action restricts the internationally community from reviewing and acting on illegal acts of international espionage and warfare, resulting precedent setting minimal repercussions for the what has been recognized to be the first use of state-sanctioned cyberwarfare. By never publically admitting involvement in the development of Stuxnet, both the United States and Israel acted unethically.


%%%%%%%%%%%%%%%%%%%%%%%%%%%%%%%%%%%%%%%
%%% Conclusion %%%
%%%%%%%%%%%%%%%%%%%%%%%%%%%%%%%%%%%%%%%
\section{Conclusion}

In conclusion, both the United States and Israel acted unethically when they participated in the development and deployment of the Stuxnet virus. Their actions violated international sovereignty law, restricting Iran's ability to develop its nuclear program independently from international intervention. Iran's strict adherence to the IAEA's inspection regime at the time of Stuxnet's initial deployment further calls to question the legitimacy of the United State's and Israel's ardent stance of deterring Iran from developing nuclear infrastructure all-together. At a low-level, Stuxnet operated by forging critical nuclear infrastructure monitoring equipment data readouts, interfering with the operations of on-site workers, resulting in civilians losing their jobs. It's intended operation was successful in destroying 984 enrichment centrifuges at the Nantez facility, hindering Iran's enrichment potential, however it's immediate effects were only temporary, while it's precedent could be acted endlessly into the future.
 

\end{multicols}

%cite all the references from the bibtex you haven't explicitly cited
%\nocite{*}

\bibliographystyle{IEEEannot}

\newpage
\bibliography{bibliography}
\end{document}
